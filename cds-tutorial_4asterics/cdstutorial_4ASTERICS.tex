\documentclass [a4paper, 12pt]{article}


\usepackage{amssymb}
\usepackage{epsfig}
\usepackage{times}
\usepackage{float}
\usepackage[usenames,dvipsnames]{color}
\usepackage{ragged2e}
\usepackage[round,sort&compress]{natbib} 
\usepackage{multicol}
\usepackage{pgfgantt}
\usepackage{hyperref}
\usepackage{graphicx}
\usepackage{subcaption}
\usepackage[ddmmyyyy]{datetime}
\setlength\columnsep{6cm}
\usepackage[top=2cm, bottom=2cm, left=2cm, right=2cm]{geometry}
\setlength{\headsep}{1cm}
\setlength{\footskip}{1cm}
\setlength{\parindent}{1cm}

\setlength{\bibsep}{-2pt}

\def\LCDM{\mbox{$\Lambda$CDM}}
\def\kms{\rm km\ s^{-1}}
\def\hkpc{\mbox{$\rm h^{-1}$kpc}}
\def\hMpc{\mbox{$\rm h^{-1}$Mpc}}
\def\hGpc{\mbox{$\rm h^{-1}$Gpc}}
\def\hMsun{\mbox{$\rm h^{-1}$M$_\odot$}}
\def\lesssim{_ <\atop{^\sim}}
\def\gtsim{_ >\atop{^\sim}}

\hypersetup{hidelinks=true}
\newcommand{\aladin}{{\textsc{A}{ladin}}}
\newcommand{\topcat}{{\textsc{Topcat}}}
\newcommand{\cassis}{{\textsc{Cassis}}}
\newcommand{\simbad}{{\textsc{SIMBAD}}}
\newcommand{\vizier}{{\textsc{v}izie\textsc{r}}}
\renewcommand{\dateseparator}{.}


\begin{document}


\begin{center}\includegraphics[width=0.3\textwidth]{../images/logo_euro.png} 
\hspace{5cm}\includegraphics[width=0.4 
\textwidth]{../images/logo_asterics.png}\\
\vspace{1.5cm}
\begin{Huge} \textbf{ASTERICS - H2020 - 653477} \end{Huge} \end{center}


\vspace{1cm}
\Huge
\begin{center}
\bf An introduction to the \\ CDS services and tools\\
\end{center}


\vspace{1cm}
\large
\begin{center}
Ada Nebot Gomez-Moran, Mark Allen and the CDS team\\
\textit{Centre de Donn\'ees astronomiques de Strasbourg, France}
\end{center}
\vspace{0.5cm}
\begin{center}
Jenny G. Sorce\\
\textit{updated with a new template for ASTERICS with the new CDS 
portal, new 
plots}\\
\vspace{0.5cm}
Katharina A. Lutz\\
\textit{updated to \aladin\ v10}
\end{center}
\vspace{0.5cm}
\begin{center}
last update \today
\end{center}


\vspace{3.5cm}
Template by Jenny G. Source


\newpage
\normalsize
\vfill
\tableofcontents
\vfill

\newpage

\justify
\section{Introduction}

The CDS harbours three major programs, accessible through the CDS portal:\\
\begin{itemize}
\item \includegraphics[width=0.1 \textwidth]{../images/logo_simbad.png} 
\textbf{\simbad}: The astronomical database \simbad\ contains more than 
8 million objects. For each object it provides basic measurements (type, 
coordinates, proper motion, radial velocity, spectral type, distance, 
magnitude), cross-correlations and bibliography.
\item \includegraphics[width=0.08  \textwidth]{../images/logo_vizier.png}  
\textbf{\vizier}: The \vizier\ catalog service provides access to about 
15,000 catalogs, being the most complete library of published astronomical 
data tables available online.
\item \includegraphics[width=0.1  \textwidth]{../images/logo_aladin.png} 
\textbf{\aladin}: The interactive sky atlas \aladin\ allows to 
visualize astronomical images and to superimpose entries from different 
catalogs and databases. It allows to visualize \simbad\ and \vizier\ 
information and distributed archives and databases as well as to upload own 
tables or images. There are two versions of \aladin: Aladin desktop and Aladin 
lite which runs in the browser.
\end{itemize}

In addition, the CDS has been offering a cross-match service 
(X-match \includegraphics[width=0.03 
\textwidth]{../images/logo_cds_xmatch.png} ) since November 2011. 

\section{Goal of this tutorial}

This tutorial shows how to use the CDS tools to gather information on 
specific astronomical objects. We will:
\begin{itemize}
\item Search for information on NGC4039 in the CDS portal
\item Search for data on NGC4039 in \aladin\
\item Compare the coverage of Sky Surveys and select interacting 
galaxies that have SDSS and GALEX data
\end{itemize}

\section{Search for information on NGC4039 in the CDS Portal}

Open the CDS Portal 
\hyperref[http://cdsportal.u-strasbg.fr/]{\textcolor{blue}{http://cdsportal.u-strasbg.fr/}}
and make a query for `NGC4039'.
The result provides an overview of the information and data available for 
this object in the 3 CDS services: \simbad, \aladin\ and \vizier: 

\begin{figure}[H]
\center
\includegraphics[width=0.5  \textwidth]{../images/cdsportal_search_ngc4039.jpg}
\caption{Query for NGC4039 in the CDS Portal}
\label{fig:cdsportal1}
\end{figure}

\subsection{\simbad\ - Identifiers, Basic Measurements and links to the 
Bibliography}

\begin{figure}[H]
    \center
    \includegraphics[width=1  
    \textwidth]{../images/cdsportal_object-information_ngc4039.jpg}
    \caption{Result of the query for NGC4039 in the CDS Portal}
    \label{fig:cdsportal2}
\end{figure}

\begin{itemize}
    \item Click on \textbf{More info in Simbad} to see the full \simbad\ 
information on this object in a new tab.
\begin{figure}[H]
    \center
    \includegraphics[width=1  
    \textwidth]{../images/simbad_section_basic-data.jpg}
    \caption{Result of the query for NGC4039 in \simbad.}
    \label{fig:simbad}
\end{figure}
    \item Note the list of \textbf{Other object types}. These types are 
drawn for the literature and are stored in \simbad\ using a 
hierarchical classification scheme. The full list of Object Types can 
be found here: 
\hyperref[http://simbad.u-strasbg.fr/simbad/sim-display?data=otypes]
{\textcolor{blue}{http://simbad.u-strasbg.fr/simbad/sim-display?data=otypes}}.
The main Object Type of NGC4039 is \textbf{Galaxy in Pairs of Galaxies 
(GiP)}.
    \item Use the {\bf parents} button \includegraphics[width=0.07 
\textwidth]{../images/simbad_button_parents.png} to identify the name 
of the galaxy pair. Sorting by the number of references 
\includegraphics[width=0.04  \textwidth]{../images/simbad_sort_references.jpg} 
can help 
bring out the most important ones. 
    \item Follow the link to the \simbad\ entry of the Antennae galaxy pair 
\includegraphics[width=0.07 
\textwidth]{../images/simbad_antennae-galaxt-pair_link.png}. There 
click on the {\bf children} button \includegraphics[width=0.07 
\textwidth]{../images/simbad_button_children.png} to identify the name 
of the two galaxies making up the galaxy pair. 
    \item Visit the \simbad\ entry of the interaction partner of NGC\,4039 and 
use the {\bf References} section to find the earliest listed reference in the 
literature to this object.
\begin{figure}[H]
    \center
    \includegraphics[width=1  
    \textwidth]{../images/simbad_section_references.jpg}
    \caption{References section}
    \label{fig:simbad}
\end{figure}

    \item Return to the CDS portal.
\end{itemize}




\subsection{\aladin\ - Images}


\begin{figure}[H]
\center
\includegraphics[width=1  
\textwidth]{../images/cdsportal_image-information_ngc4039.jpg}
\caption{Result of the query for NGC4039 in the CDS Portal}
\label{fig:cdsportal3}
\end{figure}

\begin{itemize}
\item A DSS (Digitized Sky Survey) image of NGC4039 is shown.
\item Note that you can zoom on the image by scrolling your mouse. 
\item Other images can be displayed by selecting them in the left 
column. 
\item Restricted searches on wavelength and resolution are possible by 
ticking/unticking the boxes in that same column.
\end{itemize}


\subsection{\vizier\ - Catalogs}


\begin{figure}[H]
\center
\includegraphics[width=1  
\textwidth]{../images/cdsportal_catalogue-information_ngc4039.jpg}
\caption{Result of the query for NGC4039 in the CDS Portal}
\label{fig:cdsportal4}
\end{figure}

\begin{itemize}
\item The list of catalogs is sorted by \textbf{Popularity} but can 
also be sorted by number of rows (\textbf{\#rows}), \textbf{sky 
fraction} or \textbf{year}. 
\item Note that restrictions and filtering can be applied by clicking 
on the left or right columns according to your wills.
\item Selecting a catalog, it is then possible to visualize it (either 
quickly or in \vizier), to plot it and/or to send it to the \aladin\ 
tool.
\begin{figure}[H]
\center
\includegraphics[width=1  \textwidth]{../images/cdsportal_table_usno.jpg}
\caption{Selection of a \vizier\ Catalog on the CDS Portal.}
\label{fig:cdsportal5}
\end{figure}
\item Actually the Antennae is listed in the Arp Atlas of Peculiar 
Galaxies, make a search for `Arp' into the \textbf{Search} box.

\begin{figure}[H]
\center
\includegraphics[width=0.6  
\textwidth]{../images/cdsportal_search-table_arp.jpg}
\caption{Looking for `Arp' in the \vizier\ catalogs.}
\label{fig:arp}
\end{figure}
\item Note that there are two tables for Webb 1996. Select the 
\textbf{arplist} 
table by clicking on it and then on \includegraphics[width=0.05  
\textwidth]{../images/logo_vizier.png}. This sends to the \vizier\ detailed 
query 
page.
\item Make a first query on this table by clicking on \textbf{submit}. 
Examine the output as html. Go back to the previous page
\item Modify the query preferences to add extra coordinate columns in 
J2000 decimal degrees and to obtain the whole catalog:
\begin{itemize}
    \item Remove the restriction on searching only around `NGC4039' by clicking 
    on the \textbf{Clear} button below \textbf{Target Name (resolved by 
        Sesame) or Position} at the top of the page. 
    \item In \textbf{Preferences} on the left side of the page, add extra 
    coordinate columns in decimal degrees by ticking \textbf{J2000} and 
    \textbf{decimal} boxes.
    \item Change the maximum (\textbf{max}) number of rows to 
    \textbf{unlimited} and tick \textbf{All columns} to get the whole 
    catalog.
    \begin{figure}[H]
        \center
        \includegraphics[width=0.2  
        \textwidth]{../images/vizier_preferences.jpg}
        \caption{\textbf{Preferences} panel.}
        \label{fig:pref}
    \end{figure}
    \item Submit again.
\end{itemize}

\item When satisfied, click on \textbf{Save in CDSportal}, then click 
on the \textbf{Save} button. The file is now saved in your personal 
user space on the CDS portal.
\item Click on \textbf{Go to MyData} and download a copy in 
\textbf{VOTable} format on your desktop. This file will be used later 
in the tutorial.
\begin{figure}[H]
\center
\includegraphics[width=1  \textwidth]{../images/cdsportal_mydata.jpg}
\caption{Personal User Space on the CDS Portal.}
\label{fig:download}
\end{figure}
\end{itemize}



\section{Search for data on NGC4039 in \aladin}

Open \aladin\ with at least 1GB memory allocated to the save virtual 
machine. To do so use the following command line: \texttt{java -Xmx1024m -jar 
Aladin.jar}

Aladin offers two ways to retrieve data: through the \textsc{Data Tree} on the 
left hand side of the \aladin\ window and through the 
\textsc{Server Selector} window, which can be opened via \textbf{File 
$\rightarrow$ Open server selector...} or with \textbf{CTRL + l}. 

Data sets in the \textsc{Data Tree} are colour coded in green or orange 
depending on whether or not they are available in the region currently visible 
in the main viewing window. 

\begin{figure}[H]
    \centering
    \begin{subfigure}[H]{0.4\textwidth}
        \center
        \includegraphics
            [width=0.4\textwidth]
            {../images/aladin_data_discovery_tree_open.png}
        \caption{The \textsc{Data Tree}: green entries are available for the 
        sky region currently visible in the main viewing window (not shown).}
        \label{fig:data_disc_tree}
    \end{subfigure}
    \centering
    \begin{subfigure}[H]{0.4\textwidth}
        \center
        \includegraphics[width=1 
        \textwidth]{../images/aladin_server_selector.png}
        \caption{The \textsc{Server Selector} window: here \simbad\ has been 
        queried for galaxies within 14\,arcmin around NGC\,4093.}
        \label{fig:server_selector}
    \end{subfigure}
    \caption{The two entry points to retrieving data in Aladin: The 
    \textsc{Data Tree} (left) and the \textsc{Server Selector} (right).}
\end{figure}

\begin{itemize}
    \item Start by typing "NGC4039" in the \textbf{Command} line 
\includegraphics[width=0.2\textwidth]{../images/aladin_command_ngc4039.png} 
and press \textbf{Enter}. In the main viewing window the coloured DSS image of 
the Antennae galaxies appears. As for the \aladin\ lite window in the CDS 
portal you can zoom in and out by scrolling your mouse. 

    \item Make a contour map of the image using the \textbf{cont} button 
\includegraphics[width=0.03  \textwidth]{../images/aladin_button_contours.png} 
next to main viewing window. Increase the number of contours to better 
represent the image. 
    
    \item Overlay a \simbad\ plane showing only the galaxies by selecting 
the \includegraphics[width=0.055  \textwidth]{../images/logo_simbad.png} tab  
in 
the 
\textsc{Server selector} and choosing \textbf{Galaxy} as the \textbf{Display 
filter} (see Figure~\ref{fig:server_selector}).

    \item Change the colour of the \simbad\ plane using the button 
\textbf{Properties} 
\includegraphics[width=0.03\textwidth]{../images/aladin_button_properties.png}  

    \item Using the \textbf{Select} tool \includegraphics[width=0.03  
\textwidth]{../images/aladin_button_select.png}, select some of the \simbad\ 
points and note that these are displayed as a table below the image. This 
window can be detached with the icon \includegraphics[width=0.035  
\textwidth]{../images/aladin_button_detach-table.png}. Note that the data point 
belonging to a row in the table blinks in the main viewing window when 
hoovering over the table row with the mouse.

    \item In the \textsc{Data Tree}, go to \textbf{Images $\rightarrow$ 
Infrared $\rightarrow$ 2MASS} and load the 2MASS colour image. If the box next 
to the DSS image is ticked, you can use the opacity slider 
\includegraphics[width=0.035  
\textwidth]{../images/aladin_button_opacity.png} next to the 2MASS plane in the 
image stack to change its opacity and thus compare it to the 
DSS image. Note that the different layers in the stack are displayed in the 
main window as if you are looking through the stack from top to bottom. Hence, 
if  you wanted to overlay the DSS image on the 2MASS image, you would have to 
change their order in the stack by drag-and drop. 

\begin{figure}[H]
    \center
    \includegraphics[width=1  
    \textwidth]{../images/aladin_load_2mass-rgb-image.png}
    \caption{Loading the 2MASS colour image from the \textsc{Data Tree}. }
    \label{fig:aladin_load_2mass}
\end{figure}

    \item Now add more colour images form the \textsc{Data Tree}, for example 
the allWISE (infrared) and the XMM Newton (X-rays) colour images. Compare the 
images in a number of different ways:
    \begin{itemize}
        \item Multiview: \textbf{View $\rightarrow$ Create one view per image}, 
        or via the \textbf{multiview} icon at the bottom left of the \aladin\ 
        image window \includegraphics[width=0.1 
        \textwidth]{../images/aladin_button_multiview.jpg}.
        \item Align and scale all images by using the \textbf{match} icon below 
        the image window \includegraphics[width=0.04  
        \textwidth]{../images/aladin_button_match-views.jpg}
        \item Transparency overlays: return to single view mode. Change the 
        transparency of planes in the stack with the opacity sliders         
        \includegraphics[width=0.04  
        \textwidth]{../images/aladin_button_opacity.png} as before. Remember  
        that you can move the location of the planes in the stack and thus 
        change the order in which the image are shown.
    \end{itemize}

    \item Search for more data from the VO on NGC4039:
    \begin{itemize}
        \item To restrict the collections that the \textsc{Data Tree} is 
        showing, use the \textbf{Select} line below the 
        \textsc{Data Tree} \includegraphics[width=0.1  
        \textwidth]{../images/aladin_select_transient.png}. For example when 
        searching for transients within 
        the Antennae galaxies you might restrict the search to "transient" and 
        find that the Palomar Transient Factory photometric catalogue has data 
        for transient events in the area. Load the catalogue and find the data 
        points in the main viewing window. 
        \item More elaborate filters can be created using the \textbf{Data 
        Discovery Tree Filter} window, accessed through 
        \includegraphics[width=0.04  
        \textwidth]{../images/aladin_button_filtertree.png} next to the 
        \textbf{Select} line.
        \item As before, entries in the \textsc{Data Tree}, which are coloured 
        in green, have data available in the region currently visible in the 
        main viewing window.  
    \end{itemize}
    \item Now delete all the planes in your stack before continuing. This is 
    not mandatory but it will free some useful memory space and allow you to 
    proceed more easily with the next steps.
\end{itemize}    
\begin{figure}[H]
\center
\includegraphics[width=0.7  
\textwidth]{../images/aladin_results_cdstutorial-sec4.png}
\caption{Results of all the above mentioned steps.}
\label{fig:aladinNGC4039}
\end{figure}


\section{Compare the coverage of Sky Surveys and select interacting 
galaxies that have SDSS and GALEX data}

Many large sky surveys in \aladin\ are stored in the \textbf{HiPS} format
(Hierarchical Progressive Survey, see 
\hyperref[http://aladin.u-strasbg.fr/hips/]{\textcolor{blue}
	{http://aladin.u-strasbg.fr/hips/}}),
which allows for easy access, browsing and visualisation of image and catalogue 
data. To describe (non-trivially shaped) regions on the sky \textbf{MOC} 
(Multi-Order Coverage) maps are used. In the following, we will make use of the 
advantages of these two data structures to easily asses which galaxies in the 
Arp catalogue of peculiar galaxies have been observed by SDSS and GALEX. If you 
would like to have an overview of currently available HiPS, have a look at 
\hyperref[https://aladin.u-strasbg.fr/hips/list]{\textcolor{blue}
	{https://aladin.u-strasbg.fr/hips/list}}. You can load any of the in 
\aladin\ by entering their base URL in the \textbf{Command} line.  

\begin{itemize}
    \item Select the \textbf{SDSS 9 colored} (\textbf{Image $\rightarrow$ 
Optical $\rightarrow$ SDSS}) and \textbf{GALEX All Sky Imaging Survey 
colored} (\textbf{Image $\rightarrow$ UV $\rightarrow$ GALEX}) surveys in the 
\textsc{Data Tree} and load both the imaging data (tick progressive 
\includegraphics[width=0.1\textwidth]{../images/aladin_load_progessive.png}) 
and the MOC of the survey (tick coverage
\includegraphics[width=0.1\textwidth]{../images/aladin_load_coverage.png}). For 
the moment make the two MOC planes invisible by clicking on their opacity slider
\includegraphics[width=0.04\textwidth]{../images/aladin_button_opacity.png}.
\begin{figure}[H]
\center
\includegraphics[width=0.7  
\textwidth]{../images/aladin_load_sdss-image-moc.png}
\caption{Selecting the SDSS 9 coloured survey and MOC}
\label{fig:aladinselect}
\end{figure}

    \item Turn on the coordinate \textbf{grid} 
\includegraphics[width=0.035\textwidth]{../images/aladin_button_grid.jpg}, zoom 
out and use 
the \textbf{pan} tool 
\includegraphics[width=0.035\textwidth]{../images/aladin_button_pan.png} to 
explore the whole sky.

    \item Now turn the MOC planes back on (click again on the opacity slider).
Zoom onto the edges of the surveys and note the way the MOC represents the 
coverage of the surveys.

    \item Calculate the intersection of the coverage maps of the SDSS and 
GALEX surveys using in the menu the item \textbf{Coverage $\rightarrow$ 
Logical operations} or the MOC button 
\includegraphics[width=0.035\textwidth]{../images/aladin_button_moc.png} on the 
right of the main viewing window. 

\begin{figure}[H]
\center
\includegraphics[width=0.4 \textwidth]{../images/aladin_moc_intersection.jpg}
\caption{Building the intersection of the coverage maps}
\label{fig:intersec}
\end{figure}

    \item Load the full Webb 1996 Arp catalog that you saved earlier. You 
can do this in one of the following three ways:
    \begin{itemize}
        \item \textbf{File $\rightarrow$ Load local file... }.
        \item Drag and drop on the images.
        \item Broadcast the catalogue from the \vizier\ page that shows the 
        HTML version of the table or \textbf{MyData} area of the CDS Portal. To 
        do so click on \includegraphics[width=0.035\textwidth]
        {../images/vizier_button_broadcast.png} 
        or \includegraphics[width=0.035\textwidth]
        {../images/cdsportal_button_broadcast.png} respectively and allow the 
        browser to connect to the SAMP hub. For \vizier\ you additionally have 
        to click on the newly appeared grey \textbf{Broadcast} button. Now the 
        data appears in \aladin. 
    \end{itemize}
Alternatively you can download the catalogue from \vizier\ through the 
\textsc{Data Tree}, enter "Arp Webb" in the \textbf{Select} line below the 
\textsc{Data Tree}, select the "arplist" table and \textbf{Load} the 
\textbf{Whole data}. 

    \item Filter the catalog to select only the sources that fall within 
the SDSS+GALEX MOC: \textbf{Coverage $\rightarrow$ Filter a table by 
MOC...}: 425 sources are selected. Note that you can also filter a table by MOC 
when loading the table initially (instead of loading the entire table). 
\begin{figure}[H]
\center
\includegraphics[width=0.4 
\textwidth]{../images/aladin_moc_filter-tab-by-moc.jpg}
\caption{Filtering a catalog by a MOC}
\label{fig:filtermoc}
\end{figure}

    \item Visualize the brightest ($<$9~mag) galaxies of the selected 
sources by extracting small images from the SDSS survey:

    \begin{itemize}
        \item Select the brightest galaxies using the \textbf{filter} tool  
        \includegraphics[width=0.04\textwidth]
        {../images/aladin_button_filter.png}: select the 
        \textbf{Show brightest stars} predefined filter and edit it with the 
        \textbf{Advanced mode} to select object with magnitude below 9. Note 
        that the column is automatically identified with the Unified Content 
        Descriptor `~phot.mag* '.
        \item Make sure that only the MOC filtered catalogue is active in the 
        stack and visible in the main viewing window (or other sources may also 
        be filtered). An easy way to ensure that is to delete every catalogue 
        not useful any more.
        \item Click on \textbf{Apply} and then \textbf{Export} to create a new 
        plane consisting only of sources selected by the filter. There are 7 
        sources with magnitudes below 9.
    \end{itemize}
\begin{figure}[H]
\center
\includegraphics[width=0.4 \textwidth]{../images/aladin_filter_mag.jpg}
\caption{Filtering a catalog by magnitude}
\label{fig:filtermag}
\end{figure}
\item Make thumbnails of the selected brightest sources: \textbf{Tool 
$\rightarrow$ Thumbnail view generator...}, set the thumbnail size to 
11\,arcmin. 
\begin{figure}[H]
\center
\includegraphics[width=0.6\textwidth]{../images/aladin_thumbnails.png}
\caption{Thumbnails}
\label{fig:thumbnails}
\end{figure}
\end{itemize}



\section{Collect information on a sample of galaxies using \aladin\ 
[Optional]}

Use an \aladin\ script to obtain DSS and SDSS image with HST, Chandra, ESO 
observation log overlays for each of the selected bright galaxies.

\begin{itemize}
\item Copy the script Arp\_script.ajs from the url 
\begin{tiny}\textcolor{white}{\url{http://cds.unistra.fr/tutorials/CDS-tutorial/Arp_script.ajs}}
\end{tiny} \hspace{-7.7cm} 
\textcolor{blue}{http://cds.unistra.fr/tutorials/CDS-tutorial/Arp\_script.ajs}
to your computer. The content of the script is shown in Figure 
\ref{fig:script}
\item Create a folder called `Arp' and edit the script Arp\_script.ajs 
to insert a path in order to save the output files, e.g. 
$\sim$/Desktop/Arp.
\item Open the \aladin\ macro controller and load the script:\\
- \textbf{Tools $\rightarrow$ Macro Controller} then \textbf{File 
$\rightarrow$ Load script}\\
- or cut and paste the script into the top panel of the \textsc{Macros} 
window.
\item Select all the sources in the bright galaxy catalog:\\
- Right click on the plane and \textbf{Select all objects in the 
selected planes}\\
- In the \textsc{Macros} window: \textbf{File $\rightarrow$ Use 
selected plane sources as params}.\\
- Note how the catalog columns are shown as parameters which can be 
referred to as \$1, \$2, etc within the script.
\item Click on the first row of the parameters table and execute the 
script for this row: \textbf{Exec current params}.
\item Optional: add an SDSS image: remove the `\#' to enable download 
of a SDSS g-band image for each source. Note that this results in an 
\textbf{Could not find any data corresponding to your request} message 
for objects not covered by SDSS.
\item Inspect the output in the \aladin\ window and also the files 
written in the Arp folder.
\item Execute the script for all sources: \textbf{Exec all from 
current}.
\item Note that the saved stack files can simply be dragged and dropped 
into \aladin\ for inspection.
\end{itemize}

\begin{figure}[H]
\center
\includegraphics[width=0.38 
\textwidth]{../images/aladin_macrocontroller_cdstutorial.jpg}\hspace{0.3cm}\includegraphics[width=0.6
\textwidth]{../images/aladin_script_arp.jpg}
\caption{\textsc{Macros} window \& Arp\_script.ajs }
\label{fig:script}
\end{figure}

%\bibliographystyle{unsrtnat} % Use the "unsrtnat" BibTeX style for 
%formatting the Bibliography
%\bibliography{Bibliography}


\end{document}


