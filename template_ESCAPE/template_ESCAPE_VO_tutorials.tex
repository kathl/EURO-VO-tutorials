\documentclass [a4paper, 12pt]{article}

\usepackage{amssymb}
\usepackage{epsfig}
\usepackage{times}
\usepackage{float}
\usepackage[usenames,dvipsnames]{color}
\usepackage{ragged2e}
\usepackage[round,sort&compress]{natbib}
\usepackage{multicol}
\usepackage{pgfgantt}
\usepackage{hyperref}
\usepackage{graphicx}
\usepackage{subcaption}
\usepackage[ddmmyyyy]{datetime}
\usepackage[top=2cm, bottom=2cm, left=2cm, right=2cm]{geometry}
\usepackage{../template_ESCAPE/escape_title}


\begin{document}
%%%%%%%%%%%%%%
% Title page %
%%%%%%%%%%%%%%
% First provide initial author and their affiliation as well as the
% title of the tutorial
\author{Jean Toutlemonde \& Jane Doe \\           % Authors
The great space institute }       % Affiliation
\title{Our awesome VO tutorial}   % Tutorial title
\makeescapetitle
\newpage


%%%%%%%%%%%%%%%%%%%%
% Tutorial history %
%%%%%%%%%%%%%%%%%%%%
% For every notable step forward of this tutorial, provide the name of
% the authors and their contribution here to track the development of the
% tutorial. This is always followed by \addescapehistory, which will add
% the information to the file.
\author{Jean Toutlemonde \& Jane Doe}        % Author
\title{Initial development of the tutorial}  % Contribution
\addescapehistory
\author{Erika Mustermann \& John Doe}
\title{updated for the new software package}
\addescapehistory
% and so on and so forth
\newpage


%%%%%%%%%%%%%%%%%%%%%
% Table of contents %
%%%%%%%%%%%%%%%%%%%%%
% Add a table of contents
\tableofcontents
\newpage


%%%%%%%%%%%%%%%%%%%%%%%%%%%%%
% Main body of the tutorial %
%%%%%%%%%%%%%%%%%%%%%%%%%%%%%
% Main body of the tutorial begins here. Generally the tutorials contain
% an introduction, which details among other things the goals of this tutorial
% and provide information on which software packages are required. Otherwise
% the style is free and up to you.
\section{Introduction}


\end{document}
